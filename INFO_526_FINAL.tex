% Options for packages loaded elsewhere
\PassOptionsToPackage{unicode}{hyperref}
\PassOptionsToPackage{hyphens}{url}
%
\documentclass[
]{article}
\title{COVID-19 Vaccination Availability by Socio-Economic Status}
\author{Shane Freeborn - INFO 526}
\date{}

\usepackage{amsmath,amssymb}
\usepackage{lmodern}
\usepackage{iftex}
\ifPDFTeX
  \usepackage[T1]{fontenc}
  \usepackage[utf8]{inputenc}
  \usepackage{textcomp} % provide euro and other symbols
\else % if luatex or xetex
  \usepackage{unicode-math}
  \defaultfontfeatures{Scale=MatchLowercase}
  \defaultfontfeatures[\rmfamily]{Ligatures=TeX,Scale=1}
\fi
% Use upquote if available, for straight quotes in verbatim environments
\IfFileExists{upquote.sty}{\usepackage{upquote}}{}
\IfFileExists{microtype.sty}{% use microtype if available
  \usepackage[]{microtype}
  \UseMicrotypeSet[protrusion]{basicmath} % disable protrusion for tt fonts
}{}
\makeatletter
\@ifundefined{KOMAClassName}{% if non-KOMA class
  \IfFileExists{parskip.sty}{%
    \usepackage{parskip}
  }{% else
    \setlength{\parindent}{0pt}
    \setlength{\parskip}{6pt plus 2pt minus 1pt}}
}{% if KOMA class
  \KOMAoptions{parskip=half}}
\makeatother
\usepackage{xcolor}
\IfFileExists{xurl.sty}{\usepackage{xurl}}{} % add URL line breaks if available
\IfFileExists{bookmark.sty}{\usepackage{bookmark}}{\usepackage{hyperref}}
\hypersetup{
  pdftitle={COVID-19 Vaccination Availability by Socio-Economic Status},
  pdfauthor={Shane Freeborn - INFO 526},
  hidelinks,
  pdfcreator={LaTeX via pandoc}}
\urlstyle{same} % disable monospaced font for URLs
\usepackage[margin=1in]{geometry}
\usepackage{color}
\usepackage{fancyvrb}
\newcommand{\VerbBar}{|}
\newcommand{\VERB}{\Verb[commandchars=\\\{\}]}
\DefineVerbatimEnvironment{Highlighting}{Verbatim}{commandchars=\\\{\}}
% Add ',fontsize=\small' for more characters per line
\usepackage{framed}
\definecolor{shadecolor}{RGB}{248,248,248}
\newenvironment{Shaded}{\begin{snugshade}}{\end{snugshade}}
\newcommand{\AlertTok}[1]{\textcolor[rgb]{0.94,0.16,0.16}{#1}}
\newcommand{\AnnotationTok}[1]{\textcolor[rgb]{0.56,0.35,0.01}{\textbf{\textit{#1}}}}
\newcommand{\AttributeTok}[1]{\textcolor[rgb]{0.77,0.63,0.00}{#1}}
\newcommand{\BaseNTok}[1]{\textcolor[rgb]{0.00,0.00,0.81}{#1}}
\newcommand{\BuiltInTok}[1]{#1}
\newcommand{\CharTok}[1]{\textcolor[rgb]{0.31,0.60,0.02}{#1}}
\newcommand{\CommentTok}[1]{\textcolor[rgb]{0.56,0.35,0.01}{\textit{#1}}}
\newcommand{\CommentVarTok}[1]{\textcolor[rgb]{0.56,0.35,0.01}{\textbf{\textit{#1}}}}
\newcommand{\ConstantTok}[1]{\textcolor[rgb]{0.00,0.00,0.00}{#1}}
\newcommand{\ControlFlowTok}[1]{\textcolor[rgb]{0.13,0.29,0.53}{\textbf{#1}}}
\newcommand{\DataTypeTok}[1]{\textcolor[rgb]{0.13,0.29,0.53}{#1}}
\newcommand{\DecValTok}[1]{\textcolor[rgb]{0.00,0.00,0.81}{#1}}
\newcommand{\DocumentationTok}[1]{\textcolor[rgb]{0.56,0.35,0.01}{\textbf{\textit{#1}}}}
\newcommand{\ErrorTok}[1]{\textcolor[rgb]{0.64,0.00,0.00}{\textbf{#1}}}
\newcommand{\ExtensionTok}[1]{#1}
\newcommand{\FloatTok}[1]{\textcolor[rgb]{0.00,0.00,0.81}{#1}}
\newcommand{\FunctionTok}[1]{\textcolor[rgb]{0.00,0.00,0.00}{#1}}
\newcommand{\ImportTok}[1]{#1}
\newcommand{\InformationTok}[1]{\textcolor[rgb]{0.56,0.35,0.01}{\textbf{\textit{#1}}}}
\newcommand{\KeywordTok}[1]{\textcolor[rgb]{0.13,0.29,0.53}{\textbf{#1}}}
\newcommand{\NormalTok}[1]{#1}
\newcommand{\OperatorTok}[1]{\textcolor[rgb]{0.81,0.36,0.00}{\textbf{#1}}}
\newcommand{\OtherTok}[1]{\textcolor[rgb]{0.56,0.35,0.01}{#1}}
\newcommand{\PreprocessorTok}[1]{\textcolor[rgb]{0.56,0.35,0.01}{\textit{#1}}}
\newcommand{\RegionMarkerTok}[1]{#1}
\newcommand{\SpecialCharTok}[1]{\textcolor[rgb]{0.00,0.00,0.00}{#1}}
\newcommand{\SpecialStringTok}[1]{\textcolor[rgb]{0.31,0.60,0.02}{#1}}
\newcommand{\StringTok}[1]{\textcolor[rgb]{0.31,0.60,0.02}{#1}}
\newcommand{\VariableTok}[1]{\textcolor[rgb]{0.00,0.00,0.00}{#1}}
\newcommand{\VerbatimStringTok}[1]{\textcolor[rgb]{0.31,0.60,0.02}{#1}}
\newcommand{\WarningTok}[1]{\textcolor[rgb]{0.56,0.35,0.01}{\textbf{\textit{#1}}}}
\usepackage{graphicx}
\makeatletter
\def\maxwidth{\ifdim\Gin@nat@width>\linewidth\linewidth\else\Gin@nat@width\fi}
\def\maxheight{\ifdim\Gin@nat@height>\textheight\textheight\else\Gin@nat@height\fi}
\makeatother
% Scale images if necessary, so that they will not overflow the page
% margins by default, and it is still possible to overwrite the defaults
% using explicit options in \includegraphics[width, height, ...]{}
\setkeys{Gin}{width=\maxwidth,height=\maxheight,keepaspectratio}
% Set default figure placement to htbp
\makeatletter
\def\fps@figure{htbp}
\makeatother
\setlength{\emergencystretch}{3em} % prevent overfull lines
\providecommand{\tightlist}{%
  \setlength{\itemsep}{0pt}\setlength{\parskip}{0pt}}
\setcounter{secnumdepth}{-\maxdimen} % remove section numbering
\ifLuaTeX
  \usepackage{selnolig}  % disable illegal ligatures
\fi

\begin{document}
\maketitle

\#\#Data Sets

\begin{Shaded}
\begin{Highlighting}[]
\FunctionTok{library}\NormalTok{(usmap)}
\FunctionTok{library}\NormalTok{(ggplot2)}
\FunctionTok{library}\NormalTok{(tidyverse)}
\end{Highlighting}
\end{Shaded}

\begin{verbatim}
## -- Attaching packages --------------------------------------- tidyverse 1.3.1 --
\end{verbatim}

\begin{verbatim}
## v tibble  3.1.6     v dplyr   1.0.7
## v tidyr   1.1.4     v stringr 1.4.0
## v readr   2.1.1     v forcats 0.5.1
## v purrr   0.3.4
\end{verbatim}

\begin{verbatim}
## -- Conflicts ------------------------------------------ tidyverse_conflicts() --
## x dplyr::filter() masks stats::filter()
## x dplyr::lag()    masks stats::lag()
\end{verbatim}

\begin{Shaded}
\begin{Highlighting}[]
\FunctionTok{library}\NormalTok{(ggthemes)}

\NormalTok{county\_vax }\OtherTok{\textless{}{-}} \FunctionTok{read.csv}\NormalTok{(}\StringTok{"data/COVID{-}19\_Vaccinations\_County.csv"}\NormalTok{)}
\NormalTok{cases\_by\_county\_21 }\OtherTok{\textless{}{-}} \FunctionTok{read.csv}\NormalTok{(}\StringTok{"data/us{-}counties{-}2021.csv"}\NormalTok{)}
\NormalTok{cases\_by\_county\_22 }\OtherTok{\textless{}{-}} \FunctionTok{read.csv}\NormalTok{(}\StringTok{"data/us{-}counties{-}2022.csv"}\NormalTok{)}
\NormalTok{county\_codes }\OtherTok{\textless{}{-}} \FunctionTok{read.csv}\NormalTok{(}\StringTok{"data/county\_fips.csv"}\NormalTok{)}
\NormalTok{income }\OtherTok{\textless{}{-}} \FunctionTok{read.csv}\NormalTok{(}\StringTok{"data/county\_income.csv"}\NormalTok{)}
\end{Highlighting}
\end{Shaded}

The data was acquired from multiple sources. The original intent was to
look at the accessibility of COVID-19 testing sites based on location
and socioeconomic status to determine if there is a relationship between
socio-economic status of a region and the number of testing sites.
However, data on these testing sites is not easily available or
reliable, so I decided to look at the availability and accessibility of
COVID-19 vaccines. The vaccination data comes from the CDC. The cases by
county come from the New York Times live COVID-19 data tracking. A
supplementary data set was pulled that contains unique county codes that
identify each county. Finally, income data, which determines the
socio-economic status of a county, comes from the Census Survey. All
visuals were created with older data, but new data was pulled closer to
the submission of this report.

\#\#Data Exploration

\begin{Shaded}
\begin{Highlighting}[]
\NormalTok{total\_cases\_by\_state }\OtherTok{\textless{}{-}}\NormalTok{ cases\_by\_county\_21 }\SpecialCharTok{\%\textgreater{}\%}
  
  \FunctionTok{summarise}\NormalTok{(}\AttributeTok{total\_cases =} \FunctionTok{sum}\NormalTok{(cases))}

\FunctionTok{colnames}\NormalTok{(county\_vax)[}\DecValTok{4}\NormalTok{] }\OtherTok{\textless{}{-}} \StringTok{"County"}
\NormalTok{vax\_by\_county }\OtherTok{\textless{}{-}} \FunctionTok{merge}\NormalTok{(}\AttributeTok{x =}\NormalTok{ county\_vax, }\AttributeTok{y =}\NormalTok{ income, }\AttributeTok{by=} \StringTok{"County"}\NormalTok{)}

\NormalTok{vax\_by\_county }\OtherTok{\textless{}{-}}\NormalTok{ vax\_by\_county }\SpecialCharTok{\%\textgreater{}\%}
  \FunctionTok{group\_by}\NormalTok{(County, State, Estimate\_Households\_Median\_income) }\SpecialCharTok{\%\textgreater{}\%}
  \FunctionTok{summarise}\NormalTok{(}\AttributeTok{total\_vax =} \FunctionTok{sum}\NormalTok{(Administered\_1dose))}
\end{Highlighting}
\end{Shaded}

\begin{verbatim}
## `summarise()` has grouped output by 'County', 'State'. You can override using the `.groups` argument.
\end{verbatim}

\begin{Shaded}
\begin{Highlighting}[]
\NormalTok{vax\_by\_county}\SpecialCharTok{$}\NormalTok{incomerange }\OtherTok{\textless{}{-}} \FunctionTok{cut}\NormalTok{(}\FunctionTok{as.numeric}\NormalTok{(vax\_by\_county}\SpecialCharTok{$}\NormalTok{Estimate\_Households\_Median\_income), }\FunctionTok{c}\NormalTok{(}\DecValTok{0}\NormalTok{, }\DecValTok{25000}\NormalTok{, }\DecValTok{50000}\NormalTok{, }\DecValTok{75000}\NormalTok{, }\DecValTok{100000}\NormalTok{), }\AttributeTok{labels =} \FunctionTok{c}\NormalTok{(}\StringTok{"$0{-}25k"}\NormalTok{, }\StringTok{"$26{-}50k"}\NormalTok{, }\StringTok{"$51{-}75k"}\NormalTok{, }\StringTok{"$76{-}100k+"}\NormalTok{))}
\end{Highlighting}
\end{Shaded}

\begin{verbatim}
## Warning in cut(as.numeric(vax_by_county$Estimate_Households_Median_income), :
## NAs introduced by coercion
\end{verbatim}

\begin{Shaded}
\begin{Highlighting}[]
\NormalTok{vax\_by\_county }\OtherTok{\textless{}{-}} \FunctionTok{na.omit}\NormalTok{(vax\_by\_county)}
\end{Highlighting}
\end{Shaded}

\#\#Visualizations

First we want to see a broad view of COVID cases in 2021, by state to
get an idea at which states were impacted most by the pandemic.

\begin{Shaded}
\begin{Highlighting}[]
\FunctionTok{plot\_usmap}\NormalTok{(}\AttributeTok{data =}\NormalTok{ cases\_by\_county\_21, }\AttributeTok{values =} \StringTok{"cases"}\NormalTok{) }\SpecialCharTok{+}
  \FunctionTok{scale\_fill\_colorblind}\NormalTok{() }\SpecialCharTok{+}
  \FunctionTok{scale\_fill\_continuous}\NormalTok{(}\AttributeTok{name =} \StringTok{"Total COVID{-}19 Cases 2021"}\NormalTok{, }\AttributeTok{label =}\NormalTok{ scales}\SpecialCharTok{::}\NormalTok{comma) }\SpecialCharTok{+}
  \FunctionTok{theme}\NormalTok{(}\AttributeTok{legend.position =} \StringTok{"right"}\NormalTok{)}
\end{Highlighting}
\end{Shaded}

\begin{verbatim}
## Scale for 'fill' is already present. Adding another scale for 'fill', which
## will replace the existing scale.
\end{verbatim}

\includegraphics{INFO_526_FINAL_files/figure-latex/unnamed-chunk-3-1.pdf}

\begin{Shaded}
\begin{Highlighting}[]
\NormalTok{income }\SpecialCharTok{\%\textgreater{}\%}
  \FunctionTok{group\_by}\NormalTok{(State) }\SpecialCharTok{\%\textgreater{}\%}
  \FunctionTok{summarise}\NormalTok{(}\AttributeTok{avg\_income =} \FunctionTok{mean}\NormalTok{(}\FunctionTok{as.numeric}\NormalTok{(Estimate\_Households\_Median\_income))) }\SpecialCharTok{\%\textgreater{}\%}
  \FunctionTok{ggplot}\NormalTok{(}\FunctionTok{aes}\NormalTok{(}\AttributeTok{y =}\NormalTok{ avg\_income,}
             \AttributeTok{x =} \FunctionTok{reorder}\NormalTok{(State, }\SpecialCharTok{{-}}\NormalTok{avg\_income))) }\SpecialCharTok{+} 
  \FunctionTok{geom\_bar}\NormalTok{(}\AttributeTok{stat =} \StringTok{"identity"}\NormalTok{, }\AttributeTok{width =} \FloatTok{0.4}\NormalTok{) }\SpecialCharTok{+}
  \FunctionTok{theme}\NormalTok{(}\AttributeTok{axis.text.x=}\FunctionTok{element\_text}\NormalTok{(}\AttributeTok{angle=}\DecValTok{45}\NormalTok{, }\AttributeTok{hjust=}\DecValTok{1}\NormalTok{)) }\SpecialCharTok{+}
  \FunctionTok{ggtitle}\NormalTok{(}\StringTok{"Average Income by State"}\NormalTok{) }\SpecialCharTok{+}
  \FunctionTok{xlab}\NormalTok{(}\StringTok{"State"}\NormalTok{) }\SpecialCharTok{+}
  \FunctionTok{ylab}\NormalTok{(}\StringTok{"Average Income by County"}\NormalTok{) }\SpecialCharTok{+}
  \FunctionTok{scale\_y\_discrete}\NormalTok{() }\SpecialCharTok{+}
  \FunctionTok{scale\_fill\_colorblind}\NormalTok{() }\SpecialCharTok{+} 
  \FunctionTok{scale\_color\_gradient}\NormalTok{()}
\end{Highlighting}
\end{Shaded}

\begin{verbatim}
## Warning in mean(as.numeric(Estimate_Households_Median_income)): NAs introduced
## by coercion
\end{verbatim}

\begin{verbatim}
## Warning: Removed 1 rows containing missing values (position_stack).
\end{verbatim}

\includegraphics{INFO_526_FINAL_files/figure-latex/unnamed-chunk-4-1.pdf}

\begin{Shaded}
\begin{Highlighting}[]
\NormalTok{random\_sample\_data }\OtherTok{\textless{}{-}}\NormalTok{ vax\_by\_county[}\FunctionTok{sample}\NormalTok{(}\FunctionTok{nrow}\NormalTok{(vax\_by\_county), }\DecValTok{20}\NormalTok{), ]}
\NormalTok{random\_sample\_data }\SpecialCharTok{\%\textgreater{}\%}
  \FunctionTok{ggplot}\NormalTok{(}\FunctionTok{aes}\NormalTok{(}\AttributeTok{x =}\NormalTok{ County,}
             \AttributeTok{y =}\NormalTok{ total\_vax,}
             \AttributeTok{fill =}\NormalTok{ incomerange)) }\SpecialCharTok{+} 
  \FunctionTok{geom\_col}\NormalTok{() }\SpecialCharTok{+}
  \FunctionTok{xlab}\NormalTok{(}\StringTok{"County Name"}\NormalTok{) }\SpecialCharTok{+} 
  \FunctionTok{ylab}\NormalTok{(}\StringTok{"Income Range"}\NormalTok{) }\SpecialCharTok{+}
  \FunctionTok{ggtitle}\NormalTok{(}\StringTok{"Random Sample of County Income vs Vaccinations Administered"}\NormalTok{) }\SpecialCharTok{+}
  \FunctionTok{theme}\NormalTok{(}\AttributeTok{axis.text.x=}\FunctionTok{element\_text}\NormalTok{(}\AttributeTok{angle=}\DecValTok{45}\NormalTok{, }\AttributeTok{hjust=}\DecValTok{1}\NormalTok{)) }\SpecialCharTok{+}
  \FunctionTok{scale\_fill\_colorblind}\NormalTok{()}
\end{Highlighting}
\end{Shaded}

\includegraphics{INFO_526_FINAL_files/figure-latex/unnamed-chunk-5-1.pdf}

\#\#Conclusions

We can see that there is not a strong relationship between income and
vaccinations administered. The accessibility of these vaccinations may
be impacted by factors such as income and ones ability to get to a
vaccination site, however it is not the main factor that determines if
someone can get a vaccination. Measures such as average distance to
nearest vaccination clinic and how rural a city/county may make income a
strong predictor, but it is not the only predictor of vaccination
availability.

\end{document}
